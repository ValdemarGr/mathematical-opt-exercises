\documentclass{article}
\usepackage[utf8]{inputenc}
\usepackage{graphicx}
\usepackage{verbatim}
\usepackage{listings}
\usepackage{amsmath}
\usepackage{tikz}
\usepackage{rotating}
\usetikzlibrary{positioning}
\usepackage{bussproofs}
\usepackage{turnstile}
\usepackage{stmaryrd}
\usepackage{caption}
\usepackage{subcaption}

\usepackage{cancel}
\newcommand{\lnec}{\Box}

\usepackage[edges]{forest}
\usepackage{amssymb}
\usepackage{comment}

\newcommand\mymapsto{\mathrel{\ooalign{$\rightarrow$\cr%
    \kern-.15ex\raise.275ex\hbox{\scalebox{1}[0.522]{$\mid$}}\cr}}}
\definecolor {processblue}{cmyk}{0.96,0,0,0}

\title{DM872 assignment 1}
\author{sagra16 \\(Valdemar Grange - 081097-2033) }
\date{May 2020}

\begin{document}

    \maketitle

    \section{Introduction}
    In this report I will document my attempt at creating a good schedule for events.
    These events are derived from SDU IMADA/NAT events and the model is then scored in $4$ different ways.
    I have used \texttt{gurobi} as the solver for my MILP model, and a feasible solution is found given enough time.

    \section{The model}
    The model was to have some different hard-constraints as listed in the project description.
    The objective function will be discussed at the end once all the parts have been discussed.
    I have introduced some new sets to help develop this model which have reduced the set-up time of the model tremendously (about a 20x speed increase) and smaller but noticeable solve time.
    The first of these are a set $B_r$ which is a collection of busy times for a given room $B = \{ B_r \subset P \,|\, r \in R \}$, note that this includes the banned slots from \texttt{timeslots.json} and busy from \texttt{rooms.json}.
    I will also use $E_w$ to denote all the events for week $w \in P$.
    This has an enormous impact on the time spent creating constraints, as the amount of events for a specific week is substantially smaller than the whole set of events.
    Further notation will be introduced it appears.
    Finally I have chosen that my "allocation" or "activation" variable $x$ is indexed by event $e \in E$, room $r \in R$ and time $p \in P$.
    \subsection{Hard constraints}
    \subsubsection{}
    First I have the simple constraints that $x_{e,r,p}$ is boolean and all events must be scheduled.
    Additionally I introduce the function $week: E \rightarrow \mathbb{N}$ which evaluate the scheduled week for an event (from the data).
    \[x_{erp} \in \{0, 1\}\]
    \[\sum_{(h,d,w) \in P, r \in R} x_{e,r,(h,d,week(e))} = 1, \forall e \in E\]
    The constraint states that every event must be scheduled exactly once.
    \subsubsection{}
    Next is the constraint that will ensure that no two events will ever overlap in the same room.
    \[\sum_{e \in E_w} \, \sum^h_{i=max(0, h - \ell(e) + 1)} x_{e,r,(i,d,w)} \leq 1, \, \forall (h,d,w) \in P, \, \forall r \in R\]
    Here I make use of a new and unfamiliar sum $\sum^h_{i=max(0, h - \ell(e) + 1)}$.
    This sum ensures that for some time $(h,d,w)$ and all events for the week $e$, the amount of "active" events within the time-range $\{ h - \ell(e) + 1 .. h \}$ cannot be more than $1$.
    In other words, at some point this $(h,d,w)$ will be at the last scheduled time for an event $e$, at that moment and $h - \ell(e) + 1$ steps back, only one event must be scheduled (namely the event $e$).
    We will see more of this time-range sum.
    \subsubsection{}
    This constraint states that a teacher may not teach two classes at the same time.
    Or more specifically, no more than one room may ever be occupied by a class that requires the same teacher within the entire time-frame of a lecture.
    I will additionally denote teacher events by the week they are scheduled in by $de \in D_w$.
    \[\sum_{e \in de} \, \sum_{r \in R} \, \sum^h_{i=max(0, h - \ell(e) + 1)} x_{e,r,(i,d,w)} \leq 1 , \, \forall de \in D_w, \, \forall (h,d,w) \in P,\]
    \subsubsection{}
    Here I simply ensure that no events are scheduled to run during banned periods.
    Note that I use the time-range sum here, since a event potentially could start before a banned time and range over it.
    \[\sum^h_{i=max(0, h - \ell(e) + 1)}  x_{e,r,(i,d,w)} = 0 ,\, \forall e \in E_w ,\, \forall (h,d,w) \in B_r ,\, \forall r \in R\]
    \subsubsection{}
    Now I introduce the modelling of the precedence graph, which should model the precedence within a week.
    I model this by using the "activation" day.
    If the class is allocated in day $3$ its binary value will be $1$, thus the allocated day will be $1*3$.
    The difference between the in-arc vertex's day in $e_1$ and the current day in $e_2$ must be at least $1$ (in favor of the preceding vertex being larger).
    \[\hspace*{-1cm}\left(\sum_{r \in R} \, \sum_{(h,d,w) \in P} x_{e_2,r,(h,d,week(e_2)} * d \right) - \left(\sum_{r \in R} \, \sum_{(h,d,w) \in P} x_{e_1,r,(h,d,week(e_1)} * d \right) \geq 1, \, \forall e_1e_2 \in A, \, \forall e_2 \in E\]
    \subsubsection{}
    The constraint for "there is at most one event per course per day for each student", needs an introduction of a new set $H_{d,w}$.
    $H_{d,w}$ is simply a set of hours for some $d,w \in P$.
    Note that any combination of $h,d,w$ are implied as unique combinations of the original set.
    Moreover the inclusion of the course has warranted two new additions, $C$ which is the set of courses and $Q_{c_w}$ which is the set of student events for some course and week.
    $$\sum_{r \in R} \sum_{e \in q} \sum_{h' \in H_{d,w}} x_{e,r,(h',d,w)} \leq 1, \, \forall q \in Q_{c_w}, \, \forall c \in C , \, \forall (d,w) \in P$$
    \subsubsection{}
    The final hard-constraint states that no event may exceed the last hour.
    This is modelled by introducing $final_{d,w}$ which is the final hour for a given day and week (or also expressed as $max(H_{d,w})$).
    I use the time-range sum again, but here I only use the $final_{d,w}$ as my $h$.
    The constraint will ensure that no lessons are scheduled such that they exceed the final hour of the day.
    $$\sum^{final_{d,w}}_{i=max(0, final_{d,w} - \ell(e) + 1)} x_{e,r,(i,d,w)} = 0, \, \forall e \in E_w , \, \forall (d,w) \in P , \, \forall r \in R$$
    \subsection{Objectives}
    In the objectives I can give value to different assignments and formulate what better solutions are.
    \subsubsection{}
    Discomfort is minimized in regard to bad slots.
    Let $O$ be the set of $c_p$ where $p \in P$ and $c \in \mathbb{N}$, $c$ being an arbitrary weight of how "bad" a timeslot is.
    The minimum value this $c$ may take is $1$.
    $$\alpha = \sum_{(h,d,w) \in P} \, \sum_{e \in E_w} \, \sum_{r \in R} \, \left( \sum^h_{i=max(0, h - \ell(e) + 1)} x_{e,r,(i,d,w)} \right) * c_{h,d,w}$$
    In practice I have chosen the following values:
    \begin{center}
        \begin{tabular}{|c | c | c |}
            \hline
            Time & Day(s) & $c$ \\ [0.5ex]
            \hline
            8 & Mon-Fri &  5  \\
            \hline
            16 & Mon-Thu & 3  \\
            \hline
            17 & Mon-Thu & 4  \\
            \hline
            15, 16, 17 & Fri & 6  \\ [1ex]
            \hline
        \end{tabular}
    \end{center}
    \subsubsection{}
    The number of events per day should be as low as possible.
    I find the maximum lectures $ml$ of any one day for any one teacher, this should be minimized.
    An average case minimization is harder to formulate because of the requirement of linearity (a sum of squares technique would be a nice alternative).
    \[ml \geq \sum_{e \in de} \, \sum_{h' \in H_{d,w}} \, \sum_{r \in R} x_{e,r,(h',d,w)}, \, \forall de \in D_w, \, \forall (d,w) \in P\]
    Since the solver is free to assign anything to $ml$, it will assign the smallest possible value that satisfies the constraint, eg the maximum of the right hand side of $\geq$.
    \subsubsection{}
    Like the previous constraint, an average case analysis is difficult, so I settle with the above strategy.
    I employ the same strategy as above; minimization of the maximum amount of concurrent events for a student for any given time.
    \[\beta \geq \sum_{e \in q} \sum_{r \in R} \, \sum^h_{i=max(0, h - \ell(e) + 1)} x_{e,r,(i,d,w)}, \, \forall q \in Q_w , \, \forall (h,d,w) \in P \]
    \subsubsection{}
    Finally the most difficult constraint to formulate, weekly stability.
    Weekly stability in regard to events being scheduled the same day every week, I find the day where the maximum amount of events take place, I then maximize this.
    A course is $E_c = \{e \in E | e \in C_c \}$
    \[\Delta_{c,d} =  \sum_{(h,w) \in P} \, \sum_{e \in E_{c_w}} \,\sum_{r \in R}  x_{e,r,(h,d,w)}, \, \forall d \in P, \, \forall c \in C\]
    Assuming our objective is the following:
    \[\delta_c \geq \Delta_{c,d} , \, \forall d \in P\]
    There is actually a problem here; $\delta_c$ can take \textbf{any} value in $\mathbb{N}$, so maximizing this would instantly result in $\delta_c = max(\mathbb{N}) , \, \forall c \in C$.
    \\\\
    I can get around this by using activation variables $y_{c,d}$ and big $M$ (which I simply assign to $|E|$ since a there can be at minimum $1$ course with length $E$).
    The strategy is to have a $z_c$ take the maximum value of $\Delta_{c,d}, \, \forall d \in P$.
    I can achieve this by "offsetting" a $\Delta_{c,d}$ by $\Delta_{c,d} + M y_{c,d}$ resulting in a value much larger than any value $\Delta_{c,d}$ can take, I then ensure that the solver must \textbf{only} assign $|\{d \in P\}| - 1$ values of $y_{c,d} = 1$, result in the solver picking the largest $\Delta_{c,d}$ as the "non scaled".
    $z_c \leq \Delta_{c,d} + M y_{c,d}$.
    Asking the solver to maximize this will result in the solver "picking" a $y_{c,d} = 0$ that results in the maximum of $z_c$ which of course will be $z_c = max(\{\Delta_{c,d} | d \in P\})$.
    \begin{gather*}
        y_{c,d} \in \{0, 1\}\\
        M = |E|\\
        z_c \in \mathbb{N}\\
        \sum_{d \in P} y_{c,d} = |\{d \in P\}| - 1 , \, \forall c \in C\\
        z_c \leq \Delta_{c,d} + M y_{c,d} , \, \forall d \in P , \, \forall c \in C\\
    \end{gather*}
    In practice we can instead use:
    \[
        \sum_{d \in P} y_{c,d} \leq |\{d \in P\}| - 1 , \, \forall c \in C\\
    \]
    Since largest $y_{c,d}$ will be deactivated.
    I do this because $\geq$ and $\leq$ reduce the problem complexity.\\
    Note the importance of restricting events to courses and that we invert the expression to make it a minimization problem instead of maximization.

    \subsection{Putting it together}
    Note that I split the precedence graph up into a variable for readability.
    \begin{equation*}
        \hspace{-1cm}
        \begin{array}{ll@{}ll}
            \text{minimize}  & \displaystyle\alpha + ml + \beta - \sum_{c \in C} z_c &\\
            \text{subject to}& \displaystyle\sum_{(h,d,w) \in P, r \in R} x_{e,r,(h,d,week(e))} = 1,  & \, \forall e \in E\\
            & \displaystyle\sum_{e \in E_w} \, \sum^h_{i=max(0, h - \ell(e) + 1)} x_{e,r,(i,d,w)} \leq 1, & \, \forall (h,d,w) \in P, \, \forall r \in R\\
            & \displaystyle\sum_{e \in de} \, \sum_{r \in R} \, \sum^h_{i=max(0, h - \ell(e) + 1)} x_{e,r,(i,d,w)} \leq 1 , & \, \forall de \in D_w, \, \forall (h,d,w) \in P\\
            & \displaystyle\sum^h_{i=max(0, h - \ell(e) + 1)}  x_{e,r,(i,d,w)} = 0 , & \, \forall e \in E_w ,\, \forall (h,d,w) \in B_r ,\, \forall r \in R\\
            & t_e = \sum_{r \in R} \, \sum_{(h,d,w) \in P} x_{e,r,(h,d,week(e)} * d&  \\
            & \displaystyle t_{e_2} - t_{e_1} \geq 1, & \, \forall e_1e_2 \in A, \, \forall e_2 \in E\\
            & \displaystyle \sum_{r \in R} \sum_{e \in q} \sum_{h' \in H_{d,w}} x_{e,r,(h',d,w)} \leq 1, & \, \forall q \in Q_{c_w}, \, \forall c \in C , \, \forall (d,w) \in P\\
            & \displaystyle \sum^{final_{d,w}}_{i=max(0, final_{d,w} - \ell(e) + 1)} x_{e,r,(i,d,w)} = 0, & \, \forall e \in E_w , \, \forall (d,w) \in P , \, \forall r \in R\\
            & &\\
            & \displaystyle \alpha = \sum_{(h,d,w) \in P} \, \sum_{e \in E_w} \, \sum_{r \in R} \, \left( \sum^h_{i=max(0, h - \ell(e) + 1)} x_{e,r,(i,d,w)} \right) * c_{h,d,w}\\
            & \displaystyle ml \geq \sum_{e \in de} \, \sum_{h' \in H_{d,w}} \, \sum_{r \in R} x_{e,r,(h',d,w)}, & \, \forall de \in D_w, \, \forall (d,w) \in P\\
            & \displaystyle \beta \geq \sum_{e \in q} \sum_{r \in R} \, \sum^h_{i=max(0, h - \ell(e) + 1)} x_{e,r,(i,d,w)}, & \, \forall q \in Q_w , \, \forall (h,d,w) \in P\\
            & \displaystyle \Delta_{c,d} =  \sum_{(h,w) \in P} \, \sum_{e \in E_{c_w}} \,\sum_{r \in R}  x_{e,r,(h,d,w)}, & \, \forall d \in P, \, \forall c \in C\\
            & \displaystyle \sum_{d \in P} y_{c,d} = |\{d \in P\}| - 1 , & \, \forall c \in C\\
            & \displaystyle z_c \leq \Delta_{c,d} + M y_{c,d} , & \, \forall d \in P , \, \forall c \in C\\
            & &\\
            & y_{c,d} \in \{0, 1\} & \forall c \in C, \, \forall d \in P \\
            & M = |E| &\\
            & z_c \in \mathbb{N}, & \, \forall c \in C\\
            & x_{e,r,p} \in \{0,1\},&  \forall e \in E ,\, \forall r \in R ,\, \forall p \in P &
        \end{array}
    \end{equation*}

    \section{Conclusion}
    I have run the model with different parameters, and there is room for improvement on the weighting (right now there is only weighting on the "bad" slots).
    From inspection and output from the solver I have concluded that the model works and the constraints do as they are supposed to within the domain of what I was set out to solve.
    A portion of the time was spent reducing computation time and memory consumption, such that the medium sized model could run (to a point where I left scopes to let the garbage collector, collect old data).
    I have unsuccessfully run the medium sized model entirely because of memory limits, but I suspect that $64$ gigabytes of memory (maybe $32$ with swap) would do.
    The small model runs in about $10$ minutes.
    I have supplied some figures of the scheduled times for \textit{DM872}.
    In the timeslot's the first line is the event id, the second are the teachers that can teach the course and the last is the rooms that it is scheduled to.
    Notice that the events are usually scheduled tuesday and no events are scheduled at "bad" slots in appendix 2.
    The first figure shows that banned slots are also respected in appendix 1.

    \clearpage
    \section{Appendix}
    \subsection{}
    \hspace*{-2cm}\includegraphics[scale=0.39]{../images/badslot.png}

    \subsection{}
    \begin{figure}
        \vspace{-3.2cm}
        \hspace*{-5cm}
        \begin{subfigure}{.85\textwidth}
            \centering
            \includegraphics[width=1.1\linewidth]{../images/week-14.png}
        \end{subfigure}%
        \begin{subfigure}{.85\textwidth}
            \centering
            \includegraphics[width=1.1\linewidth]{../images/week-16.png}
        \end{subfigure}
        \hspace*{-5cm}
        \begin{subfigure}{.85\textwidth}
            \centering
            \includegraphics[width=1.1\linewidth]{../images/week-17.png}
        \end{subfigure}%
        \begin{subfigure}{.85\textwidth}
            \centering
            \includegraphics[width=1.1\linewidth]{../images/week-18.png}
        \end{subfigure}
        \hspace*{-5cm}
        \begin{subfigure}{.85\textwidth}
            \centering
            \includegraphics[width=1.1\linewidth]{../images/week-18.png}
        \end{subfigure}%
        \begin{subfigure}{.85\textwidth}
            \centering
            \includegraphics[width=1.1\linewidth]{../images/week-20.png}
        \end{subfigure}
        \hspace*{-5cm}
        \begin{subfigure}{.85\textwidth}
            \centering
            \includegraphics[width=1.1\linewidth]{../images/week-21.png}
        \end{subfigure}%
        \begin{subfigure}{.85\textwidth}
            \centering
            \includegraphics[width=1.1\linewidth]{../images/week-22.png}
        \end{subfigure}
    \end{figure}

\end{document}